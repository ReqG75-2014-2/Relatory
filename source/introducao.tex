\chapter[Introdução]{Introdução}
\label{chap:introducao}
	A disciplina de Requisitos, de uma maneira geral, agrupa atividades que visam obter o enunciado completo, claro e preciso dos requisitos de um produto de \emph{software}. As seguintes definições de requisitos são aplicáveis e compatíveis com a terminologia do CMMI e dos padrões IEEE \cite{eswilson}:
		\begin{itemize}
			\item{Condição ou potencialidade de que um usuário necessita para resolver um problema ou atingir um objetivo;}
			\item{Condição ou potencialidade que um sistema, componente ou produto deve possuir para que seja aceito (isto é, satisfaça a um contrato, padrão, especificação ou outro documento formalmente imposto);}
			\item{Expressão documentada dessa característica.}
		\end{itemize}
	\ \indent É importante ressaltar que os requisitos devem ser levantados pela equipe do projeto, em conjunto com representantes do cliente, usuários e, possivelmente, especialistas da área de aplicação. Ao conjunto de técnicas empregadas para levantar, detalhar, documentar e validar os requisitos de um produto, concretiza a disciplina de Engenharia de Requisitos.
	\\ \indent Mediante aos aspectos apresentados até o momento, é importante levar em consideração que os requisitos são especificados de uma maneira particular do ponto de vista da escolha de metodologia (tradicional ou adaptativa). O grupo da disciplina de 	Requisitos de \emph{Software} adotou a metodologia ágil, baseando-se no \emph{framework} SAFe, proposto por Dean Leffingwell.

	\section[Visão Geral do Relatório]{Visão Geral do Relatório}
	\label{sec:introducao_estrutura}
		Este relatório está organizado pelos seguintes capítulos:
		\begin{itemize}
			\item[\ref{chap:introducao}.]{\textbf{\nameref{chap:introducao}}: Descreve conceitos de Requisitos e seu contexto no projeto;}
			\item[\ref{chap:contexto}.]{\textbf{\nameref{chap:contexto}}: Descreve, resumidamente, o contexto da Empresa fictícia que será contratada pela equipe do projeto para desenvolvimento de uma solução de \emph{software};}
			\item[\ref{chap:justificativa}.]{\textbf{\nameref{chap:justificativa}}: Descreve os motivos pela decisão de abordagem ágil para o desenvolvimento do projeto;}
			\item[\ref{chap:processo}.]{\textbf{\nameref{chap:processo}}: Descreve o processo de Engenharia de Requisitos desenvolvido para o contexto do projeto;}
			\item[\ref{chap:elicitacao}.]{\textbf{\nameref{chap:elicitacao}}: Descreve a análise e escolha de técnicas de elicitação de requisitos para utilização no projeto;}
			\item[\ref{chap:tgr}.]{\textbf{\nameref{chap:tgr}}: Fornece dados específicos do projeto referentes ao padrão de documento Plano de Gerenciamento de Requisitos, adaptados para o contexto do projeto;}
			\item[\ref{chap:planejamento}.]{\textbf{\nameref{chap:planejamento}}: Fornece dados preliminares de planejamento no contexto do projeto;}
			\item[\ref{chap:ferramenta}.]{\textbf{\nameref{chap:ferramenta}}: Descreve a análise e escolha de ferramentas de gestão de requisitos para utilização no projeto;}
			\item[\ref{chap:consideracoes}.]{\textbf{\nameref{chap:consideracoes}}: Conclui o tema do relatório.}
		\end{itemize}