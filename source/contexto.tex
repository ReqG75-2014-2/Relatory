\chapter[Contexto da Empresa (CHAMEX)]{Contexto da Empresa (CHAMEX)}
\label{chap:contexto}
	O grupo da disciplina de Requisitos de \emph{Software} ficou responsável por trabalhar, juntamente com o grupo da disciplina de Modelagem de Processos, soluções de \emph{software} para o contexto da empresa fictícia CHAMEX. Na Seção \ref{sec:contexto_resumo}. (\nameref{sec:contexto_resumo}) é descrito brevemente o contexto desta empresa.

	\section[Resumo CHAMEX]{Resumo CHAMEX}
	\label{sec:contexto_resumo}
		\begin{enumerate}
			\item{\textbf{Objetivo da Instituição}: Auxiliar pequenas e médias empresas privadas a melhorarem a qualidade de vida dos trabalhadores. Quanto mais disposição, vitalidade e alegria entre os trabalhadores, mais resultado positivo as empresas possuem;}
			\item{\textbf{Como}: Através do Modelo de Avaliação (MOA), que busca avaliar o nível de satisfação e qualidade de vida dos trabalhadores de uma empresa. A avaliação das práticas e processos de trabalho são feitas através de questionários.
				\begin{itemize}
					\item{\textbf{Questionários}: Os questionários possuem formas e pesos específicos. Suas respostas são controladas e geram um resultado para a empresa com um conjunto de ações necessárias para a melhoria de suas práticas e processos de trabalho. São, ao todo, três questionários:
					\begin{enumerate}
						\item{Obter a visão da empresa, quanto a suas práticas e processos de trabalho com a alta administração;}
						\item{Igualmente ao anterior, contudo, realizado com cada trabalhador da empresa;}
						\item{Obter a visão do próprio avaliador sobre o que ele vê na empresa avaliada.}
					\end{enumerate}
					Os resultados da empresa também são comparados com os resultados de outras empresas do mesmo setor, gerando um \emph{ranking} de qualidade de vida de empresas contratantes.}
				\end{itemize}}
		\end{enumerate}