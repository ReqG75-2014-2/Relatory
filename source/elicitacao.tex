\chapter[Elicitação de Requisitos]{Elicitação de Requisitos}
\label{chap:elicitacao}
	A palavra elicitar remete ao significado de descobrimento. Basicamente, elicitar pode ser definido também como tornar explícito, favorecendo a obtenção do máximo de informações para o conhecimento de um determinado objeto em questão.
	\\ \indent Em termos gerais, cabe à elicitação a tarefa de identificar os fatos relacionados aos requisitos do sistema, de forma a prover o mais correto e o maior entendimento acerca do que é demandado pelo \emph{software} que está sendo concebido.
	\\ \indent Para obtenção de êxito na elicitação de requisitos de um sistema, algumas técnicas foram desenvolvidas. Contudo, deve-se levar em consideração o contexto de trabalho para a contemplação da eficiência da técnica de elicitação. A seguir, na Seção \ref{sec:elicitacao_tecnicas} (\nameref{sec:elicitacao_tecnicas}), têm-se uma abordagem de algumas técnicas de elicitação.

	\section[Técnicas de Elicitação de Requisitos]{Técnicas de Elicitação de Requisitos}
	\label{sec:elicitacao_tecnicas}
		As técnicas de elicitação de requisitos têm como propósito a dissipação das dificuldades comuns à fase de levantamento de requisitos. Partindo do principio que todas as técnicas atualmente utilizadas possuem vantagens e desvantagens, foram definidos critérios para a escolha das técnicas utilizadas pela equipe:
		\begin{itemize}
			\item{Adequação com o contexto da equipe (disponibilidade, interatividade e experiência);}
			\item{Adequação com a abordagem escolhida;}
			\item{Compatibilidade entre as técnicas escolhidas.}
		\end{itemize}
		\ \indent Baseando-se nestes critérios, foram avaliadas quatro técnicas de elicitação. A seguir, tem-se uma breve descrição da técnica, bem como uma opinião da equipe da disciplina de Requisitos de \emph{Software} sobre a mesma.
		\begin{itemize}
			\item{\textbf{Entrevista}: Técnica tradicional para o levantamento de requisitos na fase inicial do processo de obtenção de dados. É uma técnica muito útil quando o contexto permite contato restrito a encontros pontuais. Porém com base nos critérios estabelecidos, esta técnica não se adequa completamente ao contexto da equipe, pois restringe a interatividade e não usufrui da disponibilidade do time como poderia, portanto a entrevista não foi escolhida pela equipe, tendo em vista o que ela pode oferecer com relação às outras;}
			\item{\textbf{Questionários}: Técnica amplamente aplicada quando a restrição geográfica ou de tempo é uma barreira para encontros presenciais. Geralmente, traz uma visão quantitativa dos resultados e deve haver algum entendimento prévio para a elaboração das questões. Devido a abordagem utilizada, que valoriza a interação entre as pessoas, e o contexto da equipe onde barreiras geográficas e de tempo, a princípio, não seriam grandes problemas, resolveu-se descartar esta técnica como uma escolha possível para as técnicas de elicitação de requisitos que serão usadas no nosso processo de Engenharia de Requisitos;}
			\item{\textbf{Prototipagem}: Técnica muito utilizada na elicitação de requisitos, pois possibilita uma visão prática, condizente ou não com o produto final baseado no seu nível de fidelidade, que facilita a interpretação concreta dos critérios a serem atingidos para a aceitação da porção da solução na qual a técnica foi utilizada. A escolha desta técnica foi baseada principalmente na adequação à abordagem utilizada, como uma forma de melhor expressar a rastreabilidade dos requisitos através dos protótipos criados, porém, também foi levada em conta a adequação ao contexto da equipe, através da organização das reuniões periódicas da equipe;}
			\item{\textbf{\emph{Brainstorming}}: Técnica que consiste em reuniões para a geração de ideias, onde até as ideias não convencionais são encorajadas para a agregação do maior número de ideias possíveis para serem revisadas e escolhidas, favorecendo o surgimento de soluções criativas para o problema. A escolha desta técnica foi baseada no contexto da equipe, pois pode ser realizada tanto presencialmente quanto remotamente. Na abordagem escolhida, torna mais fácil a discussão e interação na elicitação dos requisitos, o que é valorizado nas metodologias ágeis e também na compatibilidade entre as técnicas escolhidas, pois as técnicas se complementam e trazem duas visões diferentes que abrangem amplamente as possíveis dúvidas quanto aos requisitos levantados.}
		\end{itemize}